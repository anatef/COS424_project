\section{Introduction}
\label{sec:intro}

The Netflix revolution is alive and kicking. 
In the past few years, the number of American Netflix subscribers has been on a constant climb, while average movie revenues has been dropping~\cite{MisixNetflix}.
According to a recent study~\cite{VarietyNetflixBandwidth}, Netflix's Video-on-demand services account for roughly 36.5\% of overall internet bandwidth, and the bandwidth consumed by YouTube is 15.6\%; overall, more than half of the internet bandwidth is consumed by these two video streaming services.
These statistics emphasize the importance of Netflix and YouTube on the wide-ranging internet footprint, and therefore these service always seek to improve the viewer's experience.
Another emerging trend of recent years is social networks such as Facebook; around 72\% of all adults are using Facebook at least once a month, and 91\% of all millennials~\cite{FacebookStats} Data from social networks is often used to create personalized services.~\cite{carmel2009personalized}.


In our project, \textit{SocialFlix}, we attempt to add a new social aspect to online Video-On-Demand services. We wish to improve the users viewing experiences and suggest them movies that they have not yet seen, but are likely to be well accepted. We analyzed a dataset called "MovieLens"~\cite{GroupLens}, which describes 5-star rating and free-text tagging activity from MovieLens~\cite{MovieLens}, an online movie recommendation service. It contains 1,000,209 anonymous ratings of approximately 3,900 movies made by 6,040 MovieLens users who joined MovieLens in 2000. Each user in the dataset has at least 20 ratings. We used a process of collaborative filtering involving matrix factorization methods, user clustering and user cosine similarity to develop a recommender system that suggest a ranked movie list to the user. The ranked list is based on predicting movies ratings, by incorporating the user past ratings as well as all the other users past ratings and information about the users and the movies. We show that by using \hl{TODO: details on how we did in the methods/best method}

