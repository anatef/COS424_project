\section{Related Work}

The MovieLens dataset has been used in many studies tackeling similar problems of ratings prediction~\cite{5575081, zhang2006learning}. It considers a stable benchmark dataset, and its publicly available. 

Developing a personalized movie reccommender system, has been the target of several recent studies. Perhaps the most famous endeavor is the \textit{Netflix Prize}, an open competition for the best collaborative filtering algorithm to predict user ratings for movies~\cite{bell2007bellkor}. In 2009, Netflix awarded a prize of 1 millior dollars to the winning group for their algorithm, which used a combination of many predcition algorithms and approaches, each highlighting the problem from a different aspect. Their work included several baseline predictors, which they claim that capture the main apsects of the data, and are equally important as the other prediction models they have used. They used the average rating for a movie, the number of people who have rated the movie and user specific effect about the tendency of the user to rate above or below the expected rating. We chose to rely greatly on their analysis in deciding about the fetures in our data that can be relevant for the ratings predictions. 

To better charactarize movie genere preference of users, we chose to rely on the work of Kim et al.~\cite{5575081} that used an older version of the MovieLens database to create a movie recommender system. In their work they represented the past ratings of each user as a weighted vector of movie genere preferences. Based on their results that showed a decrease in the average error value, we chose to extract features representing movie genres for each user in a similar manner. They chose to start with user demographic similarity to narrow down the group of users they rely on in their predictions, but we have taken a different approach of user similarities that shows promising results. 

Other notable works in the field used matrix factorization techniques such as SVD~\cite{koren2009matrix, zhang2006learning} and NMF~\cite{nguyen2012modified, zhang2006learning}, to predict movie ratings. Such techniques are capable of reducing the high-dimensionality of our data; suggesting that using such algorithms could be a good approach for our task as well. 

Most of the studies integrated user clustering, movie clustering or both, to have the prediction be based on closely realted items~\cite{bell2007bellkor, 5575081, kim2012recommender}. This approach is called neighborhood models, and within we used clustering as well as other techniques to base our predictions only on the most relevant items in our dataset.