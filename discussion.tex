\section{Discussion}
\label{sec:discussion}

In this work using collaborative filtering, i.e. using data from all the users available in the dataset to predict results for one target user, contributed to the prediction quality. Using cosine similarity to predict the actual movie rating achieved good results in terms of the average RMSE; suggesting that the extracted features are indeed associated with ratings characteristics of users. An important observation is that using the 6 most similar users was sufficient to minimize the RMSE in all the tested users, even though these are only 0.1\% of all the users. Adding bigger number of similar users seemed to induce noise to the system rather than improve the prediction. This can further explain why the k-means clustering, which included approximately 200 users in each cluster, performed poorly for some of the tested users.

Nevertheless, the clustering results provide interesting insights on the strength of collaborative filtering with user similarity approach. Evidently higher number of ratings doesn't improve prediction quality, suggesting that small ratings number is sufficient to the construction of similar users set. Thus, we can use our system even for users with a very short ratings history. Quite surprisingly, ratings standard deviation also doesn't seem to play a role in prediction quality, even though one might expect that smaller standard deviation will result in better predictions. One possible explanation is that while some users rate all their movies in similar scores, and others use a larger range of scores, both types of users rate closely to users with similar features. This suggest that our extracted features model both users types, making our approach insensitive to the range of scores used by the user. It could also be that the number of representative users we used here is not sufficient to witness the trends evident in the ratings number and the ratings standard deviation.
As noted in Section~\ref{sec:results}, our method does somewhat better for users with higher average rating. It could be that some users tend to rate only the movies they liked, and therefore when trying to predict the ratings only for the movies they have previously rated (the ones we have the "ground truth" values for testing our prediction), the task is easier as these ratings will always be high. An interesting expansion to this work would be using unsupervised learning approaches to try to predict the ratings for movies the target user has not previously rated. We don't see any preference to using certain length of the recommendation list, i.e. the NDCG@k is not significantly better for any k in particular, and it seems that the approach either succeeds for a user in all k, or performs poorly in all of them. This suggest that some users do rate in a similar manner to users from the same cluster, while others exhibits behavior of swimming against the current, of maybe they are hipsters that hates being part of the crowd. To find such users and model them better we may need to use additional user features currently not available in the MovieLens dataset.

Using matrix factorization techniques, we chose to answer the following, slightly different, binary question: what VOD services would really want to know is whether a user will like a particular movie or not. This type of algorithms enables us to answer this question without any feature extraction, but basing our prediction solely on the users-movies ratings matrix. In some cases such meta-data on the users and movies is not available to the VOD service. Both SVD and NMF show good results of AUC $>$ 0.7. SVD performs here slightly better, but not significantly, and both methods use was supported in previous studies, so it is hard to determine which one is better for the task. An interesting follow-up to this work would be answering a larger question: whether the user has rated the movie or not. To answer this we will translate all the existing ratings to 1 and will then try to predict the existence of an "interaction", a user-movie pair with some recorded rating. This is very similar to our homework 3 assignment and if the time was in our hands, we would love to apply some of the methods we used there, to our movie recommendation system.

To conclude, in this paper we present a pipeline for predicting movie ratings for a target user, and provide a list of recommended movies based on these predictions. The experimental results show that some of the methods achieve high accuracy, and some features effect more than others on the prediction. A study such as this will pave the path to a more personalized movie recommendation system, thus improving the user experience accounted for this tremendous flow of data, world-wide.
